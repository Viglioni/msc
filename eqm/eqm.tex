% Created 2021-04-11 Sun 13:36
\documentclass[a4paper,12pt]{article}
\usepackage[utf8]{inputenc}
\usepackage[T1]{fontenc}
\usepackage{graphicx}
\usepackage{grffile}
\usepackage{longtable}
\usepackage{wrapfig}
\usepackage{rotating}
\usepackage[normalem]{ulem}
\usepackage{amsmath}
\usepackage{textcomp}
\usepackage{amssymb}
\usepackage{capt-of}
\usepackage{hyperref}
\usepackage{fullpage}
%
% American mathematical society libs
%

% imports
\usepackage{amsmath,amssymb,latexsym,amsfonts,amsthm, mathtools}

% definitions
\newtheorem{theorem}{Theorem}[section]
\newtheorem{lemma}{Lemma}[section]
\newtheorem{proposition}{Proposition}[theorem]
\newtheorem{corollary}{Corollary}[theorem]

\theoremstyle{definition}
\newtheorem{definition}{Definition}[section]
\newtheorem{example}{Example}[section]
\newtheorem{remark}{Remark}[section]

%
% Commands and aliases
%

% Complex numbers set
\newcommand{\C}{\mathbb{C}}

% Real numbers set
\newcommand{\R}{\mathbb{R}}
\newcommand{\Rn}{\mathbb{R}^n}


% Rational numbers set
\newcommand{\Q}{\mathbb{Q}}

% Integer numbers set
\newcommand{\Z}{\mathbb{Z}}

% Natural numbers set
\newcommand{\N}{\mathbb{N}}

% Prime numbers set
\renewcommand{\P}{\mathbb{P}}

% Ring of integers set
\renewcommand{\O}{\mathcal{O}}
\newcommand{\Ok}{\mathcal{O}_K}

% Ring of integers set
\newcommand{\Id}{\mathfrak{I}}

% Canonical basis
\newcommand{\Cb}{\mathcal{C}}


% Definition
\newcommand{\defsym}{\vcentcolon =}

% Id Est i.e.
\newcommand{\ie}{\textit{i.e.}}

% Maximum real subfield
\newcommand{\maxrs}{\Q(\zeta_{p} + \zeta_{p} ^{-1})}

% K_\R inner space
\newcommand{\krspace}{(K_\R,\langle{\cdot,\cdot}\rangle_\tau)}
\usepackage[utf8]{inputenc}
\usepackage[T1]{fontenc}
\usepackage{graphicx}
\usepackage{grffile}
\usepackage{longtable}
\usepackage{wrapfig}
\usepackage{rotating}
\usepackage[normalem]{ulem}
\usepackage{amsmath}
\usepackage{textcomp}
\usepackage{amssymb}
\usepackage{capt-of}
\usepackage{todonotes}
\usepackage{fullpage}
% remove links'styles
\usepackage{hyperref}
\hypersetup{pdfborder = 0 0 0}

\author{Candidate: Laura Viglioni \\ Supervisor: Prof. Dr. Ricardo Dahab}
\date{March 12, 2021}
\title{A study of some practical impacts of twisted embeddings in lattice-based cryptography}
\hypersetup{
 pdfauthor={Candidate: Laura Viglioni \\ Supervisor: Prof. Dr. Ricardo Dahab},
 pdftitle={A study of some practical impacts of twisted embeddings in lattice-based cryptography},
 pdfkeywords={},
 pdfsubject={},
 pdfcreator={Emacs 27.1 (Org mode 9.4.4)}, 
 pdflang={English}}
\begin{document}

\maketitle


\section{Introduction}
\label{sec:orgd5e6247}
\subsubsection*{Motivation}
\label{sec:org45744c6}
The imminent arrival of quantum computers in large scale demands new research fields on how to create quantum-resistant cryptosystems. In the last two decades, lattice-based cryptosystems have become an important field in the cryptography community, since these cryptosystems rely on mathematical problems we believe are hard, such as the Shortest Vector Problem \%[Definition \ref{definition:GapSVP-gap-shortest-vector-problem}]
and the Shortest Independent Vectors Problem\textasciitilde{}\cite{Peikert2017}.

Some cryptosystems of our interest in this study are the ones based on the LWE (Learning With Errors) problem 
and on the Ring-LWE (Learning With Errors over Rings)\textasciitilde{}\cite{Regev2009}.
The usual instances of the Ring-LWE are defined over the power-of-two cyclotomic number fields.

Ortiz et al.\textasciitilde{}\cite{Ortiz2021} proposed a new variety of number fields that can be used as instances of the Ring-LWE based cryptosystems using twisted embeddings,  
which are extensions of the canonical embeddings, and are as secure as the usual Ring-LWE.

The main goal of this study is to compare the implementations of usual instances of Ring-LWE and instances of the Twisted Ring-LWE, 
exploring theoretical and practical aspects.
This  includes: an analysis of both keys and messages sizes; the study of the relation between number fields and lattices with known orthonormal basis and whether it is possible to use lattices without orthonormal basis and maintain a satisfactory efficiency in converting the error samplings from \(\R^n\) points into number field elements.

\subsubsection*{Organization of this document}
\label{sec:orgbe146ec}
In Section\textasciitilde{}\ref{mathematical-background}  we present the fundamental mathematical background of the Twisted Ring-LWE schema. In Section\textasciitilde{}\ref{twisted-embeddings-and-cryptography} we discuss the impacts of the twisted embeddings in cryptography, and in Sections\textasciitilde{}\ref{objectives},  \ref{methodology} and \ref{timeline}  we discuss the objectives, methodology and an activity plan for this study.
\section{Mathematical background}
\label{mathematical-background}
\subsection{Preliminaries}
\label{sec:org23eaca5}
In this text, we  consider the Natural Numbers, i.e.,  the set of all positive integers: \(\N = \{1,2,3,\dots\}\) and the  set of all prime numbers \(\P\).

\subsection{Groups}
\label{sec:org087b7cb}

\begin{definition}
  A \textbf{group} is a set $G$, endowed with a binary operation ($\cdot$), such
  that the following properties hold:
  \begin{itemize}
  \item \textbf{Closure: } $\forall a,b \in G, \; a\cdot b \in G$.
  \item \textbf{Associativity: } $\forall a,b,c \in G, \; a\cdot(b\cdot c) = (a\cdot b)\cdot c$.
  \item \textbf{Existence of identity element: } $\exists e \in G \; ; \; \forall a \in G, \; a\cdot e = e\cdot a = a$.
  \item \textbf{Existence of inverse element: } $\forall a \in G, \; \exists b \in G \; ; \; a\cdot b = b \cdot a = e$.
  \end{itemize}
Such a group is denoted $\langle G,\cdot\rangle$ or, simply, $G$ if the operation is clear from the context.
\end{definition}

\begin{definition}
  A group is said to be \textbf{commutative} or \textbf{abelian}
  if $\forall a, b \in G, \; a\cdot b = b\cdot a$.
\end{definition}

\begin{text}
  \noindent
  A group is called \textbf{additive} if we denote its operation by $+$, its identity element by $0$, and the ``addition'' of $k$ terms $a+a+\ldots + a$ by $ka$. Likewise, a group is called  \textbf{multiplicative} if $*$ is its operation, $1$ its identity element, and $a^k=a * a * \ldots * a$. 

\end{text}

\begin{definition}
  A subset $H$ of $G$ is a \textbf{subgroup} of $\langle G,\cdot \rangle$ if it is
  closed under $\cdot$ induced. The \textbf{trivial subgroup} of any
  group is the set consisting of just the identity element.
\end{definition}

\begin{definition}
  The \textbf{order} of a group $\langle G,\cdot\rangle$ is the cardinality of the set $G$.
\end{definition}

\begin{definition}
  A subgroup $H$ of $G$ can be used to decompose $G$ in uniformly sized and
  disjoints subsets called \textbf{cosets}. Given an element $g \in G$:
  \begin{itemize}
  \item A \textbf{left coset} is defined by $gH := \{g\cdot h \; ; \; h \in H\}$.
  \item A \textbf{right coset} is defined by $Hg := \{h\cdot g \; ; \; h \in H\}$.
  \end{itemize}
\end{definition}   
\subsection{Rings and fields}
\label{sec:org6176793}
   \begin{definition}
A \textbf{ring} is a set, together with two binary operations that we denote 
$+$ and $*$, such that:
\begin{itemize}
\item $\langle R,+\rangle$ is an abelian group.
\item $*$ is associative.
\item $*$ is distributive over $+$.
\end{itemize}

A ring is denoted by $\langle R,+,*\rangle$ or, simply, $R$ if the operations are clear from the context.
\end{definition}

\begin{definition}
  A ring is said to be \textbf{commutative} if its $*$ operation is commutative.
\end{definition}

\begin{definition}
  A ring is said to be \textbf{with unity} if $*$ has an identity element. We
  shall denote it by $1$ and call it  \textbf{unity}.

\end{definition}

\begin{definition}
  A \textbf{division ring} is a ring R with unity where, $\forall r \in R, \; \exists s \in R \; ; \; r*s = 1$.
\end{definition}

\begin{definition}
  A \textbf{field} is a commutative division ring.
\end{definition}
\subsection{Number fields}
\label{sec:org558d49c}
 \begin{definition}
  For $K, L$ two fields, we denote by $L/K$ a \textbf{field extension} if  $K \subseteq L$. Then $L$ is said to be an \textbf{ extension field} over $K$, or just an \textbf{extension} over $K$.
\end{definition}


\begin{text}
  Note that in a field extension \(L/K\), \(L\) has the structure of a vector space over
  \(K\), where vector addition is in \(L\) and scalar multiplication, for \(a \in K, \; v \in L\), then $av \in L$. The dimension of \(L\) as a vector space is called
  \textbf{degree} and is denoted by \([L:K]\).

  We call a \textbf{simple extension} when the extension is generated by the
  adjunction of a single element. We denote it by $L = K(\theta)$, where
  $\theta \in L$ is the added element. The same idea is applied for rings, and it is
  denoted by $L = K[\theta]$.

\end{text}

\begin{definition}
  A field extension is called a  \textbf{number field} when it is over the rational field $\Q$. 
\end{definition}

\begin{definition}
  Let $\alpha \in L$, where $L/K$ is a field extension. We say that $\alpha$ is
  \textbf{algebraic over $K$} if $\exists p \in K[X] \;;\; p(\alpha) = 0$. $p$ is said to be
  \textbf{the minimal polynomial of $\alpha$ over $K$} denoted by $p_\alpha$. If $\alpha \in L =
  \Q(\theta)$, we simply call $\alpha$ an \textbf{algebraic number}.
\end{definition}

\begin{example}
  It is known that $\Q$ is a field. If we add $\sqrt{2}$ to the set, we
  can build a new field adding also all the powers and multiples of
  $\Q$. This new field is denoted by $\Q(\sqrt{2})$, note that
  $\sqrt{2}$ is algebraic and its minimal polynomial $p_{\sqrt{2}} = x^2-2$. All
  elements of $\Q(\sqrt{2})$ are of the form $\{a+b\sqrt{2} \;|\; a,b \in
  \Q\}$ and one of its basis is $\{1, \sqrt{2}\}$, so it has degree is
  $2$.
\end{example}

\begin{example}
  If we add $\sqrt[3]{2}$ to $\Q$ instead, its elements would have the
  form $\{a + b\sqrt[3]{2} + c\sqrt[3]{4} \;|\; a,b,c \in \Q\}$, so one of
  its basis is $\{1 ,\sqrt[3]{2} ,\sqrt[3]{4}\}$, $p_\alpha = x^3 - 2$ and its degree
  is $3$.
\end{example}

\begin{example}[\cite{Ortiz2021}, Cyclotomic number field]\label{example:cyclotomic-number-field}
  A number field of particular interest is $\Q(\zeta_m)$, the $m$-th cyclotomic field,
  where $\zeta_m = e^{2\pi i /m}$ is a primitive $m$-th root of unity for any
  integer number $m \geq 1$. The degree of $\Q(\zeta_m)$ is $\phi(m)$, where $\phi(\cdot)$
  denotes Euler’s totient function. The minimal polynomial of $\zeta_m$, called
  the $m$-th cyclotomic polynomial, is $\Phi_m(x) = \prod_{k \in \Z_{m}^*}{(x-\zeta^k_m)}$, where $\Z^*_m$ denotes the group of invertible elements in $\Z/m\Z$.
\end{example}

\begin{example}[\cite{Ortiz2021}, Maximal real subfield]
  \label{example:maximal-real-subfield}
  The number field $\Q(\zeta_m + \zeta_m^{-1}) \subset \R \cap \Q(\zeta_m)$ is the maximal real subfield of $\Q(\zeta_m)$ and has degree $\phi(m)/2$ if $m \geq 3$.
\end{example}

\begin{theorem}
  [\cite{stewart2002}, p.40] If $K$ is a number field, then $K = \Q(\theta)$ for some
  algebraic number $\theta \in K$, called primitive element.
\end{theorem}

\begin{text}
  We conclude that \(\{1, \theta, \theta^2, \dots , \theta^{n-1}\}\) is a basis for the vector
  space \(K = \Q(\theta)\) over \(\Q\). Note that we can represent an number \(a \in K\) as a linear combination of \(\theta\), \emph{i.e} \(a = \sum^n_{i=0}{a_i\theta^i}\) or as a polynomial \(a(x) = \sum^n_{i=0}{a_ix^i}\).

\end{text}

\begin{definition}
A number $\alpha$ is said to be an \textbf{algebraic integer} if $ p \in \Z[X] \;;\; p(\alpha) = 0$. The set of all algebraic integers of $K$ forms a ring called \textbf{ring of integers} of $K$ and is denoted by $\Ok$.
\end{definition}

\begin{definition}[Fixed field by involution]\label{definition:fixed-field-by-involution}
  A map $f: K \rightarrow K$, where $K$ is a number field, is called \textbf{involution}
  of $K$ if $\forall a,b \in K \; f(a+b) = f(a) + f(b) \; f(a \cdot b) = f(a) \cdot f(b)$ and
  $f(f(a)) = a$. The subfield $F = \{a \in K \;;\; f(a) = a\}$ is called a \textbf{fixed field by
    involution} of $K$.
\end{definition}

\subsection{The inner product space \emph{H}}
\label{sec:org9db062f}
\begin{definition}
  \label{definition:the-h-space}
  Let $r,s,n \in \Z_+$ such that $n = r + 2s > 0$. The space $H \subset \C^n$ is defined
  as:
  \begin{equation*}
    H = \{(a_1,\dots, a_r, b_1,\dots, b_s, \overline{b_1}, \dots, \overline{b_s}) \in \C^n\},
  \end{equation*}
  where $a_i \in \R, \; \forall i \in \{1,\dots,r\}$ and $b_j \in \C, \; \forall \; j \in \{1,\dots,
  s\}$. For all $x = \left(x_1, \dots, x_n\right), y = \left(y_1, \dots, y_n\right) \in H$ the space
  $H$ is endowed with inner product $\langle {x,y} \rangle_H$ defined as:
  \begin{equation*}
    \langle {x,y} \rangle_H = \sum_{i=1}^n{x_i \overline{y_i}} = \sum_{i=1}^r{x_i y_i} + \sum_{i=1}^s{x_{i+r} \overline{y_{i+r}}} + \sum_{i=1}^s{\overline{x_{i+r}} y_{i+r}}.
  \end{equation*}

  The $\ell_2$-norm and infinity norm of any $x \in H$ are defined as $\|x\| =
  \sqrt{\langle{x,x}\rangle_H}$ and $\|x\|_\infty = \max{\{ |x_i| \}}_{i=1}^n $.
\end{definition}

It can be proven that \(H\) and \(\R^n\) are isomorphic.
\subsection{Lattices}
\label{sec:org09a1056}
\subsubsection{Basic definitions}
\label{sec:org4ca835e}
\begin{definition}
  A \textbf{lattice} $\Lambda \subset \R^n$ is a discrete subgroup of the additive group $\R^n$.  In other words, given $m$ linear independent vectors in $\R^n$, the set
  $\{v_1, v_2, \dots, v_m\}$ is called a \textbf{basis} for $\Lambda$ and the lattice may be defined
  by:

  \begin{equation*}
    \Lambda := \left\{x = \sum_{i=1}^m{\lambda_i v_i} \in \R^n \; ; \; \lambda_i \in \Z\right\}.
  \end{equation*}

  That is, any $\lambda \in \Lambda$ can be written as $\lambda = Mv$, where $M$ is the
  \textbf{generator matrix} of $\Lambda$ where each row is a vector from the basis and
  $v \in \Z^n$. When $M$ is a square matrix, we say $\Lambda$ is a
  \textbf{full-rank lattice}.
\end{definition}


\begin{text}
  Since space \(H\) (\ref{definition:the-h-space}) is isomorphic to \(\R^n\), all definitions above can be switched from \(\R^n\) to \(H\) without any loss of generality.

\end{text}

\begin{definition}
  The \textbf{minimum distance} of a lattice $\Lambda$ is the shortest nonzero vector
  from $\Lambda$, given some norm, \textit{i.e.}:
  $$
  \lambda_1(\Lambda) \defsym \min_{0 \ne v \in \Lambda}{\|v\|}.
  $$

  We define $\lambda_m$ as the set of $m \in \N$  linear independent vectors of $\Lambda$
  such that the largest vector from $\lambda_m$ is smaller or equal to the biggest vector of any linearly independent set of length $m$ in $\Lambda$. We usually use
  $\lambda_n$, where $n$ is the size of the basis of $\Lambda$ and we call them
  \textbf{shortest independent vectors} of $\Lambda$.
\end{definition}

\begin{definition}\label{definition:gram-matrix}
  Let $\Lambda$ be a lattice and $M$ its generator matrix. The matrix $G = MM^T$ is called the \textbf{Gram matrix} for $\Lambda$.
\end{definition}
\subsubsection{Lattice problems}
\label{sec:org55f3e6c}
    \begin{definition}[\cite{Peikert2017}, Definition 2.8, Gap Shortest Vector
  Problem]
  \label{definition:GapSVP-gap-shortest-vector-problem}
For an approximation factor $\gamma  = \gamma(n) \geq 1$, the $GapSVP_\gamma $ is: given a lattice
$\Lambda$ and length $d > 0$, output \textbf{YES} if $\lambda_1(\Lambda) \leq d$ and \textbf{NO} if
$\lambda_1(L) > \gamma d$.  
\end{definition}

\begin{definition}[\cite{Peikert2017}, Definition 2.8, Shortest Independent
  Vectors Problem]
  \label{definition:SIVP-shortest-independent-vector-problem}
  For an approximation factor $\gamma = \gamma(n) \geq 1$, the $SIVP_\gamma$ is: given a lattice $\Lambda$, output $n$ linearly independent lattice vectors of length at most $\gamma(n) \cdot \lambda_n(\Lambda)$.
\end{definition}
\subsection{Learning problems}
\label{sec:org25b0143}
I this section we will describe some problems that are believed to be hard and used in cryptography. 
\subsubsection{Learning from Parity}
\label{sec:org68b1a9c}
\begin{definition}
  \label{definition:LFP-learning-from-parity-problem}
  Given $m$ vectors uniformly chosen  $a_i \gets \Z^n_2$ and some $\epsilon \in [0,1]$, we
  define the problem \textbf{Learning from Parity (LFP)} as:

  Find $s \in \Z^n_2$ such that, for $i \in \{1,\dots,m\}$
     $$ \langle{s, a_i}\rangle \; \approx_\epsilon \; b_i \;\; (mod\; 2). $$

     In other words, the equality holds with probability $1 - \epsilon$.

\end{definition}

\subsubsection{Learning with Errors}
\label{sec:orgb9da43d}

\begin{definition}\label{definition:LWE-learning-with-errors-problem}
  \textbf{Learning with Errors (LWE)} is a generalization of LFP (\ref{definition:LFP-learning-from-parity-problem}) with two new parameters $p \in \P$ and $\chi$ a probability distribution on $\Z_p$ so that we have:
\[
  <s, a_i> \; \approx_\chi \; b_i \pmod p \;\;\; \text{or} \;\;\; <s, a_i> + \; e_i =  b_i \pmod p ,
    \]
     where $a_i \gets \Z^n_p$ uniformly and $e_i \gets \Z$ according to $\chi$. We denote
     it by $LWE_{p,\chi}$.

\end{definition}


\begin{theorem}[\cite{Regev2009}, Theorem 1.1]
  Let $n$, $p$ be integers and $\alpha \in (0, 1)$ be such that $\alpha p > 2\sqrt{n}$. If
  there exists an efficient algorithm that solves $LWE_{p \Psi_\alpha}$ then there
  exists an efficient quantum algorithm that approximates the decision version
  of the shortest vector problem ($GAP_{SVP}$ \ref{definition:GapSVP-gap-shortest-vector-problem}) and the
  shortest independent vectors problem (SIVP \ref{definition:SIVP-shortest-independent-vector-problem}) to within
  $\tilde{O}(n/\alpha)$ in the worst case,  where $\Psi_\beta$ is defined as:
  $$
  \forall r \in [0,1), \; \Psi_\beta(r) \defsym \sum_{k=-\infty}^\infty{\frac{1}{\beta} . \exp{\left( -\pi \left( \frac{r-k}{\beta} \right)^2 \right)}}.
  $$
\end{theorem}
\subsection{Twisted Embeddings}
\label{sec:org803c9d7}
\subsubsection{Embeddings}
\label{sec:org280cbf1}

\begin{definition}
Let $K$ and $L$ be two number fields and a homomorphism $\phi: K \rightarrow L$. $\phi$ is said to be a \textbf{$\Q$-homomorphism} if $\phi(a) = a, \forall a \in \Q$.
\end{definition}

\begin{definition}
A $\Q$-homomorphism $\phi: K \rightarrow \C$ is called an \textbf{embedding}.
\end{definition}

\begin{theorem}
[\cite{stewart2002}, p.41] If $K$ is a number field with degree $n$ then there are
exactly $n$ embeddings $\sigma_i : K \rightarrow \C$ where by $\sigma_i(\theta) =
\theta_i$ where $\theta_i \in \C$ is a distinct zero of $K$'s
minimum polynomial.
\end{theorem}


\begin{definition}[Trace and Norm]\label{definition:trace-and-norm}
  Let $K/\Q$ a number field with degree $n$, $x \in K$ be an element of a number field and $\{\sigma_i\}_{i=1}^n$ the $n$ embeddings. The elements $\{\sigma_i(x)\}_{i=1}^n$ are called \textbf{conjugates} of
  $x$ and we define the \textbf{norm} $N(x)$ of $x$  and \textbf{trace} $Tr(x)$ of $x$,
  respectively:
  \[
    N(x) = \prod_{i=1}^n{\sigma_i(x)} \;,\;   Tr(x) = \sum_{i=1}^n{\sigma_i(x)}.
  \]

\end{definition}

\begin{theorem}[\cite{stewart2002}, p.54]
  For any $x \in K$, we have $N(x), Tr(x) \in \Q$. If $x \in \Ok$, we have $N(x),
  Tr(x) \in \Z$.
\end{theorem}

\begin{definition}
An \textbf{integral basis} is a $\Z$-basis for a ring of integers. 
\end{definition}

\begin{definition}[\cite{Peikert2017}, Section 2.3.2]
  An \textbf{integral Ideal} $\Id \subset \Ok$ is a  nontrivial additive subgroup that
  is also closed under multiplication by $\Ok$, \textit{i.e.}, $r \cdot a \in \Id$ for
  any $r \in \Ok$ and $a \in \Id$. Any ideal $\Id$ is a free $\Z$-module of rank
  $n$, \ie, it is the set of all $\Z$-linear combinations of some basis
  $\{b_1,\dots,b_n\} \subset \Id$  of linearly independent (over $\Z$) elements $b_i$.
\end{definition}

\begin{definition}[\cite{Peikert2017}, Section 2.3.2]
  A \textbf{fractional ideal} $\Id \subset K$ is a set such that $d\Id \subset \Ok$ is an
  integral ideal for some $d \in \Ok$
\end{definition}

\begin{definition}[\cite{Peikert2017}, Section 2.3.3]
  For any fractional ideal $\Id \subset K$, its \textbf{dual ideal} is defined as
  $\Id^\vee \defsym \{ a \in K \;;\; Tr(a\Id) \subset \Z \}$. An important canonical
  fractional ideal in a number field K is the \textbf{codifferent ideal}
  $\Ok^\vee$, \ie, the dual ideal of the ring of integers:
  $\Ok^\vee \defsym \{ a \in K \;;\; Tr(a \Ok) \subset \Z \}$.
\end{definition}


\begin{definition}
  Let $\{\sigma_i\}_n$ be the $n$ embeddings of a number field $K$. Let $r$ the number of embeddings with real images and $2s$ the non-real ones; then
  $r + 2s = n$. The pair $\left(r,s\right)$ is called \textbf{signature} of $K$.
\end{definition}

   \begin{definition}\label{definition:canonical-embedding}
The homomorphism $\sigma: K \rightarrow \R^r \times \C^s$, where $(r,s)$ is the signature of $K$, is the \textbf{canonical embedding} and is defined by:

\[
\sigma(x) = \left(\sigma_1(x), \dots , \sigma_r(x), \sigma_{r+1}(x), \dots, \sigma_{r+s}(x) \right).
\]

Note that we could rewrite the canonical embedding as $\sigma : K \rightarrow \R^n$,
\[
\sigma(x) = \left( \sigma_1(x), \dots , \sigma_r(x), \Re(\sigma_{r+1}(x)), \Im(\sigma_{r+1}(x)), \dots,
  \Re(\sigma_{r+s}(x)), \Im(\sigma_{r+s}(x)) \right).
\]


From now on, when working on $\Rn$, we will denote it simply by:

\[
\sigma(x) = \left( \sigma_1(x), \dots , \sigma_r(x), \sigma_{r+1}(x), \dots, \sigma_{r+2s}(x) \right).
\]

\end{definition}
\subsubsection{Algebraic lattices}
\label{sec:orgc66ac54}

\begin{theorem}[\cite{stewart2002}, p.155]\label{theorem:algebraic-lattice}
Let $\{\omega_1,...,\omega_n\}$ be an integral basis of $K$. The $n$ vectors $v_i = \sigma(\omega_i)
\in \R^n$ are linearly independent, so they define a full rank algebraic lattice
$\Lambda = \Lambda(\Ok) = \sigma(\Ok)$.
\end{theorem} 
The generator matrix of \(\Lambda = \sigma(\Ok)\) is defined by

\begin{equation}
  \label{definition:gen-matrix-alg-lattices}
  \begin{pmatrix}
    \sigma_1(\omega_1) & $\dots$ &  \sigma_{r+2s}(\omega_1) \\
    & \vdots & \\
    \sigma_1(\omega_n) & $\dots$ & \sigma_{r+2s}(\omega_n) \\
  \end{pmatrix}.  
\end{equation}

\begin{remark}\label{remmark:lattices-number-field-correspondence}
  The canonical embedding creates the correspondence between a point $\lambda \in \Lambda \subset
  \R^n$ of an algebraic lattice (Theo.~\ref{theorem:algebraic-lattice}) and an integer in $\Ok$:

  Let $\lambda$ be a point of a lattice $\Lambda$:

  \begin{align*} 
    \lambda &= (\lambda_1,\dots,\lambda_{r+2s}) \in \Lambda \\
      &= \left( \sum_{i=1}^n{z_i\sigma_1(\omega_i)} , \dots , \sum_{i=1}^n{z_i\sigma_{r+2s}(\omega_i)} \right) \\
      &= \left( \sigma_1\left(   \sum_{i=1}^n{z_i\omega_i} \right) , \dots , \sigma_{r+2s} \left( \sum_{i=1}^n{z_i\omega_i}  \right) \right), 
  \end{align*}
  where $z_i \in \Z$. Since any element $x \in \Ok$ has the form $x =
  \sum_{i=1}^n{\lambda_i\omega_i}$, we can conclude that

  \begin{equation*}
    \lambda = \left( \sigma_1(x), \dots, \sigma_{r+2s}(x) \right) = \sigma(x).
  \end{equation*}

\end{remark}
\subsubsection{Ring-LWE}
\label{sec:orga0d8d8e}

    \begin{definition}
  Let $K$ be a number field with degree $n$ and $\sigma$ an embedding. We say that a
  number $\tau \in F$, where $F$ is the fixed field by involution of $K$ (Definition \ref{definition:fixed-field-by-involution}), is \textbf{totally  positive} if $\forall i \in {1, \dots , n}, \; \sigma_i(\tau) \in \R^*_+$. 
\end{definition}

\begin{text}
  Let $K$ be a number field, $R = \Ok$ its ring of integers and $R^\vee$ the
  codifferent ideal of $K$. Let $2 \leq q \in \N$ and for any fractional ideal $\Id \subset
  K$. Also let $K_\R$ be the tensor product $K \otimes_\Q \R$, $\Id_q = \Id/q\Id$
  and $\mathbb{T} = K_\R/R^\vee$.

  The twisted embeddings can be extended from $K$ to $K_\R$ as follows [\cite{Ortiz2021},
  Section 3]: for any totally positive $\tau \in F$, the $\R$-vector space
  $\sigma_\tau(K_\R)$ is isomorphic to $H \simeq \R^n$. Consider the extension of the trace
  function $Tr_K : K \rightarrow \Q$ to $Tr_K : K_\R \rightarrow \R$. For any $\tau \in F$ totally
  positive integer, we can define the inner product as:

  $$
  \langle{a,b}\rangle_\tau \defsym \langle{\sigma_\tau(a), \sigma_\tau(b)}\rangle_H  = Tr_K (\tau a \overline{b}) , \;\; a,b \in K_\R
  $$

  By considering the inner product $\langle{a,b}\rangle_\tau$, the $\R$-vector space $K_\R$
  is an Euclidian vector space of dimension $n$ isometric to both $\left(
    H , \langle{a,b}\rangle_H  \right)$ and $\left( \Rn , \langle{a,b}\rangle  \right)$.
\end{text}

\begin{definition}[\cite{Peikert2017}, Definition 2.15, Ring-LWE Average-Case Decision]
  \label{definition:ring-lwe-decision}
  Let $\Upsilon$ be a distribution over a family of error distributions over $K_\R$.
  The \textbf{average-case Ring-LWE decision problem}, denoted $R-LWE_{q,\Upsilon}$, is to
  distinguish (with non-negligible advantage) between independent samples from
  $A_{s, \psi}$ for a \textit{random} choice of $(s,\psi) \longleftarrow U(R_q^\vee) \times \Upsilon$, and the
  same number of uniformly random and independent samples from $R_q \times \mathbb{T}$.
\end{definition}

\begin{theorem}[\cite{Peikert2017}, Corollary 5.2]
  Let $\alpha = \alpha(n) \in (0, 1)$, and let $q = q(n)$ be an integer such that $\alpha q \geq 
  2\sqrt{n}$. Then, there is \emph{a polynomial-time quantum reduction from} $SIVP_{\gamma'}$
  and $GapSVP_{\gamma'}$ \emph{to (average-case, decision)} $LWE_{q,\alpha}$.
\end{theorem}

\begin{definition}[\cite{Lyubashevsky2010}, Definition 3.2, Ring-LWE Search]
  \label{definition:ring-lwe-search}
Let $\Psi$ be a family of distributions over $K_\R$. The \textbf{search version of the $ring-LWE$ problem}, denoted $R-LWE_{q,\Psi}$, is defined as follows: given access to arbitrarily many independent samples from $A_{s,\psi}$ for some arbitrary $s \in R_q^\vee$ and $\psi \in \Psi$, find $s$.
\end{definition}

\begin{theorem}[\cite{Lyubashevsky2010}, Theorem 3.6]
  Let K be the $m^{th}$ cyclotomic number field having dimension $n = \phi(m)$ and $R =
  \Ok$ be its ring of integers. Let $\alpha < \sqrt{(\log{n})/n}$, and let $q = q(n)
  \geq 2, \; q = 1 \; (mod \; m)$ be a $poly(n)$-bounded prime such that $\alpha q \geq
  \omega(\sqrt{\log{n}})$. Then there is a polynomial-time quantum reduction from
  $\tilde{O}(n/\alpha)$-approximate $SIVP$ (or $SVP$) on ideal lattices in $K$ to
  $R-DLWE_{q,\Upsilon_\alpha}$. Alternatively, for any $l \geq 1$, we can replace the target
  problem by the problem of solving $R-DLWE_{q,D_\xi}$ given only $l$ samples,
  where $\xi = \alpha \cdot ( nl/ \log{(nl)} )^{1/4}$.
\end{theorem}
\subsubsection{Twisted embeddings}
\label{sec:org68c912f}

\begin{definition}[Twisted Embedding]
  \label{definition:twisted-embeddings}
  Given $\tau$ a totally positive number, the \textbf{$\tau$-twisted embedding}, or
  simply twisted embedding, is the monomorphism defined as
  \begin{equation*}
    \sigma_\tau(x) = \left( \sqrt{\tau_1}\sigma_1(x), \dots, \sqrt{\tau_{r+2s}}\sigma_{r+2s}(x) \right),
  \end{equation*}
where $\tau_i = \sigma_i(\tau)$.
\end{definition}
\section{Twisted embeddings and cryptography}
\label{twisted-embeddings-and-cryptography}
\subsection{Twisted Ring-LWE}
\label{sec:org81b7778}
In this section we present a variant of the Ring-LWE (Definition \ref{definition:ring-lwe-search}) using twisted embeddings (Definition \ref{definition:twisted-embeddings}).

\begin{definition}[\cite{Ortiz2021}, Twisted Ring-LWE distribution]
  \label{definition:twisted-ring-lwe}
  For a totally positive element $\tau \in F$, let $\psi_\tau$ denote an error distribution
  over the inner product $\langle{\cdot,\cdot}\rangle_\tau$ and $s \in R^\vee_q$ (the “secret”) be an
  uniformly randomized element. The \emph{Twisted Ring-LWE distribution}
  $\mathcal{A}_{s,\psi_\tau}$ produces samples of the form
  $$
  a, b = a \cdot s + e \pmod{qR^\vee} \in R_q \times K_\R/qR^\vee.
  $$
\end{definition}

Solving the Twisted Ring-LWE is as hard as solving the usual Ring-LWE as stated in Theorem \ref{theorem:twisted-ring-lwe-hardness}:

\begin{theorem}[\cite{Ortiz2021}, Theorem 1]
  \label{theorem:twisted-ring-lwe-hardness}
  Let $K$ be an arbitrary number field, and let $\tau \in F$ be totally positive.
  Also, let $(s,\psi)$ be randomly chosen from $(U(R_q^\vee)\times \Psi)$ in $(K_\R,\langle{\cdot,\cdot}\rangle_{\tau=1})$.
  Then there is a polynomial-time reduction from $\mbox{Ring-LWE}_{q,\psi}$ to $\mbox{Ring-LWE}^\tau_{q,\psi_\tau}$.
\end{theorem}
\subsection{Error sampling in rotated \(\Z^n\)-lattices}
\label{sampling-in-zn-rotated-lattices}
   \begin{text}
  In this section we present the \textit{Ortiz et al.} (\cite{Ortiz2021}, Section 8)
  variation of the cryptosystem of Lyubashevsky, Peikert, and Regev
  (\cite{LPV2013}, Section 8.2) using twisted embeddings. Let $R$ be an $m$-th
  cyclotomic ring, $p, q \in \Z$ coprime numbers and $K = \Q(\zeta_p + \zeta_p^{-1})$. The message space is defined as
  $R_p$ and it is required that $q$ be coprime with every odd prime dividing
  $m$. Consider that $\phi_\tau$ is an error distribution over $\krspace$
  and $\lfloor{\cdot}\rceil$ denotes a valid discretization to (cosets) of $R^\vee$ or $pR^\vee$.
  Also, $\hat{m} = m/2$ if $m$ is even, otherwise $\hat{m} = m$. Finally, for any
  $\overline{a} \in \Z_q$, let $[[\overline{a}]]$ denote the unique representative
  $a \in (\overline{a} + q\Z) \cap [-q/2, q/2)$, which is entry-wise extended to
  polynomials.

  \begin{itemize}
  \item \textbf{Key generation}: choose a uniformly random $a \in R_q$. Choose $x
    \longleftarrow \lfloor{\phi_\tau}\rceil$ and $e \longleftarrow \lfloor{p \cdot \phi_\tau}\rceil_{pR^\vee}$. Output $(a,b = \hat{m}\cdot(a \cdot x + e)
    \mod{qR} ) \in R_q \times R_q$ as the public key and $x$ as the secret key.
  \item \textbf{Encryption}: choose $z \longleftarrow  \lfloor{\phi_\tau}\rceil_R^\vee$, $e' \longleftarrow \lfloor{p \cdot
      \phi_\tau}\rceil_{pR^\vee}$ and  $e'' \longleftarrow \lfloor{p \cdot \phi_\tau}\rceil_{t^{-1}\mu +pR^\vee}$, where $\mu \in R_p$ is
    the word to be encrypted. Let $u = \hat{m} \cdot (a \cdot z + e') \mod{qR}$ and $v =
    z \cdot b + e'' \in R_q^\vee$. Output $(u,v) \in R_q \times R^\vee_q$.
  \item \textbf{Decryption}: Given the encrypted message $(u,v)$, compute $v - u
    \cdot x \mod{qR^\vee}$, and decode it to $d = [[v - u \cdot x]] \in R^\vee$. Output $\mu = t \cdot
    d \bmod{pR}$. 
  \end{itemize}

  In this cryptosystem, the most expensive operations to compute are the error sampling, its discretization and the polynomial multiplications. When $R$ is
  the ring of integers of the maximal real subfield
  (\ref{example:maximal-real-subfield}) $\maxrs$, the sampling of error terms can be performed directly over $(K_\R, \langle{\cdot,\cdot}\rangle_\tau)$ in the orthonormal basis while preserving the spherical format and standard deviation in respect to the corresponding distribution in $H$. The efficiency
of discrete sampling when $K = \Q(\zeta_p + \zeta_p^{-1})$ is reinforced by the fact
that the discretization in $\Z^n$-lattices is simply a coordinate-wise rounding to the nearest integer. (\cite{Ortiz2021}, Section 8).
\end{text}
\subsection{Impacts of the twisted embeddings}
\label{sec:org249acbc}

   \begin{text}
  The correspondence between a point $\lambda \in \Lambda$ of a lattice and an algebraic
  integer $x \in \Ok$ of a ring of integers (Remark \ref{remmark:lattices-number-field-correspondence}),
  \ie, $\lambda = (\sigma_1(x), \dots, \sigma_{r+2s}(x)) = \sigma(x)$, where $\sigma$ is the
  canonical embedding (Definition \ref{definition:canonical-embedding}), allow us to
  sample errors over a lattice and convert them through the embedding to the
  polynomial representation, \ie, the representation of an element of a ring of
  integers.

  This conversion is trivial when the lattices we are dealing with are rotations of
  $\Z^n$, otherwise it can be very expensive. With the canonical embedding
  (Definition \ref{definition:canonical-embedding}) we can achieve a $\Z^n$-rotated
  lattice with the cyclotomic number field with power of $2$ dimension
  (\cite{Lyubashevsky2010}, \cite{DucasDurmos2012}).

  Using the twisted embedding (Definition \ref{definition:twisted-embeddings}) we can obtain
  different lattices from the same number field:

\end{text}

   \begin{example}[\cite{Ortiz2021}, Example 3]\label{example:twisted-embeddings-generating-different-lattices-from-the-same-ring}
  Let $K = \Q(\sqrt{3}) = \{a + b\sqrt{3} \;;\; a,b \in \Q\}$ be a totally
  real number field with degree 2. It follows that the fixed field by
  involution $F=K$. For any totally positive element $\tau \in F$, consider
  the lattice $M_\tau = \Ok = \Z[\sqrt{3}]$ in the inner product space
  $(K_\R,\langle \cdot,\cdot \rangle_\tau)$. The set $\{1,\sqrt{3}\}$ in a
  $\Z$-basis of $M_\tau$ and the Gram matrix of the lattice $M_\tau$ is given by:
  \[G_\tau =
    \begin{bmatrix}
      Tr_K(\tau) & Tr_K(\tau\sqrt{3}) \\
      Tr_K(\tau\sqrt{3}) & Tr_k(3\tau)
    \end{bmatrix}.
  \]

  For example, for $\tau = 1$ \text{and} $\tau = 2 + \sqrt{3}$, the Gram matrices are
  given by:
  \[
    G_1 =
    \begin{bmatrix}
      2 & 0 \\
      0 & 6
    \end{bmatrix}
    \;\;\;and\;\;\;
    G_{2+\sqrt{3}} =
    \begin{bmatrix}
      4 & 6 \\
      6 & 12
    \end{bmatrix}.
  \]
  It can be shown that these two lattices are not equivalent.
\end{example}

\begin{text}
  Theorem \ref{theorem:ideal-lattices-doesnt-change-gaussian}), 
  Proposition \ref{prop:maximal-real-subfield-generates-orthonormal-lattice})
  and Corollary \ref{corollary:maximal-real-subfield-prime-p-greater-than-5}), bellow, show that we can build $\Z^n$-rotated
  lattices from the maximal real subfield
  (Example \ref{example:maximal-real-subfield}) using twisted embeddings, \ie, the
  errors sampled on these lattices can be trivially converted to polynomial
  representations as elements of a number field.
\end{text}

\begin{theorem}[\cite{Ortiz2021}, Theorem 5]\label{theorem:ideal-lattices-doesnt-change-gaussian}
  Let $K$ be a number field with a fixed field by the involution $F$. Consider $\tau
  \in F$ totally positive and $\Id \subset \Ok$ a fractional ideal such that $\Id$ is an
  ideal lattice in $(K_\R , \langle \cdot,\cdot \rangle_\tau )$. If $\Id$ is an orthonormal lattice, then both the format and the standard deviation of a spherical Gaussian distribution in an orthonormal basis of $\Id \subset K_\R$ are preserved when seen in the canonical basis of the space $H$ (via the twisted embedding $\sigma_\tau$).
\end{theorem}

\begin{proposition}[\cite{Ortiz2021}, Proposition 2]\label{prop:maximal-real-subfield-generates-orthonormal-lattice}
Let $p \geq 5$ be a prime number, and let $K = \maxrs$ and $\tau =
  \frac{1}{p}(1 - \zeta_p)( 1 - \zeta^{- 1}_p)$. Then $\Ok$ in $\krspace$ is an
  orthonormal lattice with basis $\Cb^\perp = \{e_1^\prime, \dots, e^\prime_n \;;\; e^\prime_n = e_n
  \;\; \text{and} \;\; e^\prime_j = e_j +  e^\prime_{j+ 1} \}$ where $\Cb = \{e_1,\dots,e_n\}$
  is the integral basis of $K$.
\end{proposition}

\begin{corollary}[\cite{Ortiz2021}, Corollary 1]\label{corollary:maximal-real-subfield-prime-p-greater-than-5}
  Let $K = \maxrs$ for $p \geq 5$ prime and let $v \in \Ok$ be a random variable
  distributed as $\psi_s^n$ in the basis $\Cb^\perp$. Then, the dstribution of $(T^{-1}
  \circ \sigma_\tau)(v)$ for $\tau = \frac{1}{p}(1 - \zeta_p)( 1 - \zeta^{- 1}_p)$, seen in the
  canonical basis of $H$, is the spherical Gaussian $\psi_s^n$.
\end{corollary}

These new constructions with a larger variety of possible rings broaden the security notions of Ring-LWE (Definitions \ref{definition:ring-lwe-search}, \ref{definition:ring-lwe-decision}), since specific rings might have specific vulnerabilities that other rings do not. It is important to remark that each number field has its own polynomial representation and, specifically, a polynomial \(f(x)\) that defines the ring we use as a parameter in the Ring-LWE cryptosystems. That said, the size of the parameters, therefore keys, encrypted messages etc, and the cost of the Ring-LWE operations depend on the polynomial representation of the ring and of \(f(x)\).

There is, though, an open question as to whether there exist other number fields that could be used build orthonormal lattices and whose polynomial arithmetic are efficient enough to be used in cryptosystems. 
\section{Objectives}
\label{objectives}
As presented in previous sections, the study of Ortiz et al.\textasciitilde{}\cite{Ortiz2021}  shows that it is possible, and as secure [Theorem \ref{theorem:twisted-ring-lwe-hardness}] as Ring-LWE [Definitions \ref{definition:ring-lwe-search}, \ref{definition:ring-lwe-decision}], to use twisted embeddings [Definition \ref{definition:twisted-embeddings}] instead of the canonical embedding [Definition \ref{definition:canonical-embedding}] to broaden the variety of rings and lattices that can be used. That is, the arithmetic involved can be done in an efficient way. Example [\ref{example:twisted-embeddings-generating-different-lattices-from-the-same-ring}] shows that, from the same ring, it is possible to generate different lattices, changing the parameters in the twisted embedding.

With the cyclotomic power-of-two number field [Example \ref{example:cyclotomic-number-field}], we can sample errors from orthonormal lattices using the canonical embedding. With the twisted embeddings we can do it with the maximal real subfield [Example \ref{example:maximal-real-subfield}] for any prime \(p\) bigger than \(3\) [Prop \ref{prop:maximal-real-subfield-generates-orthonormal-lattice}, Corollary \ref{corollary:maximal-real-subfield-prime-p-greater-than-5}]. Our objective in  this work is to validate the idea of using twisted embeddings in cryptography, explore the theoretical and the practical aspects of this proposal.

As for practical aspects, the core of this study, we want to compare  implementations  of the Twisted Ring-LWE and Ring-LWE using specific instances, \ie, maximum real subfield versus the cyclotomic power-of-two. That includes  implementations but also a search for proper sizes of  keys and  messages, in order to check the viability of this proposal not only in security terms [Theorem \ref{theorem:twisted-ring-lwe-hardness}] but also in efficiency terms.

Regarding theoretical aspects, we want to study the polynomial arithmetic of the maximal real subfield along with lattices generated by it and its properties, including the relation between the orthonormal basis and the efficient conversion between lattice points and elements of number fields [Remmark \ref{remmark:lattices-number-field-correspondence}]. Also, examine if it is possible to achieve a satisfactory efficiency with non-orthonormal basis; also, within the orthonormal context, examine whether other number fields have efficient polynomial arithmetic. And, of course, a discovery about keys and message sizes using these number fields.
\section{Methodology}
\label{methodology}
In order to achieve the objectives discussed above, we propose the following methodology:
\begin{itemize}
\item \textbf{Literature Review:} review proposals of new cryptosystems , such as \emph{NTTRU}.
\item \textbf{Theoretical experiments:} for a given number field, perform experiments using algebra libraries -- \emph{e.g.} SageMath -- to discover twist factors that enhance characteristics of the resultant lattice, such as the shortest vector. Also, perform experiments to discover orthonormal bases.
\item \textbf{Experimental outcome:} for the resultant number fields of the previous experiments, to calculate the expansion factor of a polynomial \(f(x)\) that defines the ring \(\Z[x]/f(x)\). Also,  adapt or develop algorithms for polynomial multiplication with \((O(n\log{n}))\) complexity and moderate constants.
\item \textbf{Implementation:} implement a Twisted Ring-LWE based cryptosystem.
\item \textbf{Practical experiments:} perform experiments using the implemented cryptosystem to estimate the cost in terms of clock cycles, also key and message sizes.
\end{itemize}

\section{Timeline}
\label{timeline}
\subsection{Activities}
\label{sec:org8c1c6e0}
\begin{itemize}
\item First and second semesters of 2021
\begin{itemize}
\item Study the Twisted Ring LWE problem and implementation.
\item Perform theoretical experiments with number fields, twist factors and lattices.
\item Calculate the expansion factor and adapt/develop algorithms for polynomial multiplication.
\end{itemize}
\item First and second semesters of 2022
\begin{itemize}
\item Implement a Twisted Ring-LWE based cryptosystem.
\item Compare instances of Ring LWE and Twisted Ring LWE, \ie, analyze the cryptosystem in both terms of clock cycles and key sizes.
\item Defense of dissertation.
\end{itemize}
\end{itemize}

\section*{Bibliography}
\label{bibliography}
\bibliographystyle{plain}
\bibliography{library,ic-tese-v3}
\end{document}